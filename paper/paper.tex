% Documento LaTeX com o artigo que estamos escrevendo

% Cabeçalho
% Onde a gente configura o documento
%%%%%%%%%%%%%%%%%%%%%%%%%%%%%%%%%%%%%%%%%%%%%%%%%%%%%%%
\documentclass{article}

\usepackage[brazil]{babel}


% Corpo
% Onde a gente escreve o texto
%%%%%%%%%%%%%%%%%%%%%%%%%%%%%%%%%%%%%%%%%%%%%%%%%%%%%%%
\begin{document}

\title{Análise de variação de temperatura dos últimos cinco anos}
\author{Leonardo Uieda, Yago Moreira Castro, Arthur Siqueira de Macêdo}

\maketitle

\begin{abstract}
Meu resumo legalzão.
\end{abstract}

\section{Introdução}

Isso vai ser a minha introdução.
Outra frase da introdução.

Esse já será outro parágrafo da introdução.

\section{Metologia}
\label{sec:metodos}

Aqui eu vou descrever tudo que eu fiz.
Ajustamos uma reta aos cinco últimos anos dos dados
de temperatura média mensal para cada país.
Assim calculamos a taxa de variação da temperatura recente.

A equação da reta é

\begin{equation}
T(t) = a t + b,
\label{eq:reta}
\end{equation}

\noindent
onde $T$ é a temperatura, $t$ é o tempo, $a$ é o coeficiente angular 
e $b$ é o coeficiente linear.

Utilizamos a equação \ref{eq:reta} em um código Python para fazer o ajuste da 
reta com o método dos mínimos quadrados. 
Isso está descrito na seção \ref{sec:metodos}.



\end{document}