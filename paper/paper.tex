% Documento LaTeX com o artigo que estamos escrevendo

% Cabeçalho
% Onde a gente configura o documento
%%%%%%%%%%%%%%%%%%%%%%%%%%%%%%%%%%%%%%%%%%%%%%%%%%%%%%%
\documentclass{article}

\usepackage[brazil]{babel}
\usepackage{graphicx}
\usepackage[round,authoryear,sort]{natbib}
\usepackage{notomath}

\newcommand{\Title}{Análise de variação de temperatura dos últimos cinco anos}

\input{paises.tex}



% Corpo
% Onde a gente escreve o texto
%%%%%%%%%%%%%%%%%%%%%%%%%%%%%%%%%%%%%%%%%%%%%%%%%%%%%%%
\begin{document}

\title{\Title}
\author{Leonardo Uieda, Yago Moreira Castro, Arthur Siqueira de Macêdo}

\maketitle

\begin{abstract}
Meu resumo legalzão. \Title.
\end{abstract}

\section{Introdução}

Isso vai ser a minha introdução.
Outra frase da introdução.

Esse já será outro parágrafo da introdução.

Trabalhos anteriores bem legais fizeram coisas parecidas
\citep{Hansen2010}.
Isso foi analisado primeiro por \citet{Hansen2010}.


\section{Metologia}
\label{sec:metodos}

Aqui eu vou descrever tudo que eu fiz.
Ajustamos uma reta aos cinco últimos anos dos dados
de temperatura média mensal para cada país.
Assim calculamos a taxa de variação da temperatura recente.

A equação da reta é

\begin{equation}
T(t) = a t + b,
\label{eq:reta}
\end{equation}

\noindent
onde $T$ é a temperatura, $t$ é o tempo, $a$ é o coeficiente angular 
e $b$ é o coeficiente linear.

Utilizamos a equação \ref{eq:reta} em um código Python para fazer o ajuste da 
reta com o método dos mínimos quadrados. 
Isso está descrito na seção \ref{sec:metodos}.

\section{Resultados}

Analisamos os dados de 225 países. 
Os países analisados foram: \Paises.

Os resultados da análise de variação de temperatura estão na figura \ref{fig:variacao}.

\begin{figure}[!tb]
    \centering
    \includegraphics[width=0.5\columnwidth]{../figuras/variacao_temperatura.png}
    \caption{
        Variação de temperatura média mensal dos cinco últimos anos. 
        a) Países com as cinco maiores variações de temperatura.
        b) Países com as cinco menores variações de temperatura.
    }
    \label{fig:variacao}
\end{figure}


\bibliographystyle{apalike}
\bibliography{referencias.bib}


\end{document}
